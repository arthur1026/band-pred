\section{Conclusion}
\label{sec:conclusion}
We proposed to use machine learning techniques to predict available bandwidth
in ultra-high speed networks. We trained different regression algorithms across
different different datasets, and demonstrated the feasibility of such learning
based approaches. Our estimators are robust to different probing speed,
different probing length. Although we have better performances in some
datasets, but for new network environments without training data, such methods
won't be promising.

\section{Future Work}
\label{sec:future_work}
As can be seen in Table~\ref{tab:dataset}, many probing packages are lost, and
we discarded all in-complete probing sequences. One interesting problem is how
we can take advantages of such partial information to infer available
bandwidth. We believed that the high dimensional sequences actually lay on a
much lower dimensional manifold, such partial sequence may convey enough
information to do the regression. We can use techniques like compressed
sensing\cite{donoho2006compressed} to do such inference.

We fit models to a specific network setting and environments, but real-world
network environments are complex and varying. The generalization ability of
learned models to new environments is beyond the scope of this course project,
but can be an interesting problem to investigate.

We also stated that lacking training dataset can harm the performance of
machine learning approaches. One way to mitigate such problem is adopting
online learning algorithms. Training data are fed into the currently learned
model in a streaming fashion, and the model keep updating itself based on new input data.

\subsubsection*{Acknowledgments}

We specially thank Qianwen Yin for providing datasets (both raw and smoothed)
and inspiring discussions.
