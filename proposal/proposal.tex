% !TEX TS-program = pdflatex
% !TEX encoding = UTF-8 Unicode

\documentclass[11pt]{article}
\usepackage[utf8]{inputenc}
\usepackage{graphicx}
\graphicspath{ {./figures/} }
\DeclareGraphicsExtensions{.pdf,.png,.jpg}

\usepackage{booktabs} % for much better looking tables
\usepackage{array} % for better arrays (eg matrices) in maths
\usepackage{paralist} % very flexible & customisable lists (eg. enumerate/itemize, etc.)
\usepackage{verbatim} % adds environment for commenting out blocks of text & for better verbatim
\usepackage{subfig}
\usepackage{amsmath}
\usepackage{amssymb}
\usepackage{listings}
\usepackage{color}


\definecolor{dkgreen}{rgb}{0,0.6,0}
\definecolor{gray}{rgb}{0.5,0.5,0.5}
\definecolor{mauve}{rgb}{0.58,0,0.82}

\lstset{ %
  backgroundcolor=\color{white},  % choose the background color; you must add \usepackage{color} or \usepackage{xcolor}
  basicstyle=\ttfamily\footnotesize,       % the size of the fonts that are used for the code
  breakatwhitespace=false,        % sets if automatic breaks should only happen at whitespace
  breaklines=true,                % sets automatic line breaking
  captionpos=t,                   % sets the caption-position to bottom
  commentstyle=\color{dkgreen},   % comment style
  deletekeywords={...},           % if you want to delete keywords from the given language
  escapeinside={\%*}{*)},         % if you want to add LaTeX within your code
  frame=single,                   % adds a frame around the code
  keywordstyle=\color{blue},      % keyword style
  language=MATLAB,                % the language of the code
  morekeywords={*,...},           % if you want to add more keywords to the set
  numbers=left,                   % where to put the line-numbers; possible values are (none, left, right)
  numbersep=5pt,                  % how far the line-numbers are from the code
  numberstyle=\tiny\color{gray},  % the style that is used for the line-numbers
  rulecolor=\color{black},        % if not set, the frame-color may be changed on line-breaks within not-black text (e.g. comments (green here))
  showspaces=false,               % show spaces everywhere adding particular underscores; it overrides 'showstringspaces'
  showstringspaces=false,         % underline spaces within strings only
  showtabs=false,                 % show tabs within strings adding particular underscores
  stepnumber=1,                   % the step between two line-numbers. If it's 1, each line will be numbered
  stringstyle=\color{mauve},      % string literal style
  tabsize=2,                      % sets default tabsize to 2 spaces
  title=\lstname,
  }


\title{Network Bandwith Prediction using Machine Learning Approaches}
% \subtitle{COMP 631 Course Project Proposal}
\author{Ke Wang, Lu Chen} % It's alphabetical :)
\date{\{kewang,luchen\}@cs.unc.edu}
\begin{document}
\maketitle

\section{Introduction}
\label{sec:introduction}
It is said that the proposal should not be longer than 1 page~\cite{yin2014}.
\section{Motivation}
\label{sec:motivation}
No methods deal with buffering-related noise( from bursty cross traffic and
severe interrupt coalescence, which totally distort probe streams) at
ultra-high speed networks well. Can machine learning help?

\section{Proposal}
\label{sec:proposal}

Estimating available bandwidth on ultra-high speed networks is an important yet
not well-solved problem~\cite{yin2014}. Noises coming from burst cross traffic
and severe interrupt coalescence could distort probe streams. Most established
methods handle buffering-related noises well only on low speed networks (100Mbps
or lower).

Instead of using hand-designed mechanisms, we consider using modern machine
learning approaches to automatically discover underlying patterns of such
noises, thus providing an alternative way of predicting available bandwidth on
ultra-high speed networks. Time permitting, we would like to explore different
machine learning techniques. For example, linear regression methods like
Lasso~\cite{Tibshirani1996}, deep neural networks, etc.


\section{Timeline}
\label{sec:timeline}




\bibliography{ref}
\bibliographystyle{plain}
\end{document}
